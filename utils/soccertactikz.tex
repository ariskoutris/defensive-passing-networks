\documentclass[a4paper,12pt]{ltxdoc}
\usepackage[T1]{fontenc}
\usepackage[utf8]{inputenc}
\usepackage{a4wide,multicol}

% the package
\newcommand{\TikZ}{Ti\emph{k}Z}
\newcommand{\SoccerTacTikZ}{SoccerTac\TikZ}
\usepackage{soccertactikz}
\tikzset{pitchunit=0.75ex,>=stealth}
\input{ethcolors}

\usepackage{rotating}
\usepackage{wrapfig}\setlength{\intextsep}{0ex}
\graphicspath{{./figs/}}

\usepackage[pdftex,colorlinks,
    urlcolor=red,linkcolor=red,citecolor=blue,%
    pdftitle={The SoccerTacTikZ Package},%
    pdfsubject={Diagramming association football (soccer) tactics},%
    pdfauthor={Ulrik Brandes}
   ]{hyperref}

% -----------------------------------------------------------------
\title{The \textsf{\SoccerTacTikZ} Package for \LaTeX\thanks{%
\SoccerTacTikZ\ has been developed originally for illustrations 
in a course on \emph{Soccer Analytics},
and has evolved into an attempt to define graphical conventions.}\\
\normalsize ---~version 0.9~---}
\author{Ulrik Brandes, ETH Zürich}
\date{2024-10-01}
% -----------------------------------------------------------------

\begin{document}
% =================================================================
\maketitle

% -----------------------------------------------------------------
\begin{abstract}
The \textsf{\SoccerTacTikZ} package provides macros
for association football (soccer) pitches and tactical diagrams. 
It includes standard player and interaction symbols 
and supports multiple vendor-specific coordinate systems.
\end{abstract}
% -----------------------------------------------------------------
\tableofcontents
\clearpage

\section{Introduction}
% =================================================================

The \textsf{\SoccerTacTikZ} package
was implemented to create illustrations
for \emph{Soccer Analytics}, an annual course at ETH Zurich,
but it is increasingly used for other purposes
such as theses, articles, and coaching materials.
 
Therefore, the package is now made available for general use,
and this documentation explains its underlying rationale.
As the name suggests, \textsf{\SoccerTacTikZ}
adds some football-related features to 
to the graphic might of the \TikZ~\cite{tikz} package.
As a byproduct of facilitating visual representations,
we hope to establish conventions for tactical diagrams.
A number of design choices have been made specifically
to promote accurate depictions and a standardized visual language.

Obviously, football is played between two teams, 
and we refer to them generically as the home and away team.
Since tactical analyses are usually focused on one of them,
we refer to the focal team as the home team.
The denominations home vs.\ away thus take the role of
us vs.\ them, or focal team vs.\ opposition
and should not be interpreted literally.


\section{Using the Package}
% =================================================================

The \SoccerTacTikZ\ package provides \dots

\subsection{Loading and Options}\label{sec:options}
% -----------------------------------------------------------------

To make use of \textsf{\SoccerTacTikZ},
the file |soccertactikz.sty| should be
obtained from \url{https://github.com/ubrandes-ethz/soccertactikz}
and placed in a directory
searched by the \LaTeX\ installation.
It is loaded by placing
\begin{quote}
  \cs{usepackage}\oarg{options}{\ttfamily\{soccertactics\}}
\end{quote}
in the preamble of a \LaTeX\ document. 
Several other packages will be loaded automatically,
not least because \SoccerTacTikZ\ requires \TikZ~\cite{tikz},
which has its own dependencies.

The following package options allow to set initial parameter values,
but can be adjusted separately for each diagram.
\begin{description}
\item[\tt scale] The unit of measure for pitch coordinates is the SI unit meter. \cs{scale} defaults to 1000, yielding a scale of 1:1000, so that~1m on the pitch corresponds to~1mm in the diagram. 
\end{description}
Options can be combined arbitrarily. Declaring, for instance,
\begin{quote}
  \cs{usepackage[scale=1050,XXX]\{soccertactikz\}}
\end{quote}
scales a standard pitch of 105m~length to~10cm,
and \dots



\subsection{Pitches}\label{sec:pitch}
% -----------------------------------------------------------------
% viewports, zones

Since we are taking the perspective of one of the two teams,
this team also defines a direction of play,
i.e., which of the two goals are
to be defended and attacked, respectively.

To reduce cognitive dissonance
between graphical depictions and spatial orientation on the pitch,
tactics boards are generally set up
such that the longer side is vertical.
This way, left and right retain their usual meaning,
vertical passes are indeed vertical,
and dropping midfielders are actually moving to a lower location.

\begin{multicols}{2}
\noindent
\tikz[pitchunit=1mm]\pic[green]{pitch};

\noindent
By default, pitch diagrams are therefore oriented
so that the direction of play is upward for the focal team.
If context requires, they can still be oriented horizontally,
but for the reasons stated above this is discouraged.
\begin{center}
\begin{tikzpicture}[pitchunit=0.3ex]
\pic[green,rotate=-90]{pitch};
\draw (60,28) node[right]{left wing}
      (60,-28) node[right]{right wing};
\end{tikzpicture}
\end{center}
In addition, 
pitch dimensions, markings, and other elements
are drawn to scale to convey an accurate sense of space
no matter the actual size of the diagram.
\end{multicols}


\subsection{Players}
% -----------------------------------------------------------------

While any symbol could be used to represent players,
we assign the two most common ones to players of the 
home~\tikz\pic[home]{player}; and away~\tikz\pic[away]{player}; team.

These symbols can be colored to match
shirts~\tikz\pic[home,purple40]{player};,
shorts~\tikz\pic[home,purple40,shorts=purple]{player};,
and shirt numbers~\tikz\pic[home,purple40,shorts=purple,number=8]{player};.

Since there 

with and without numbers

in \tikz\pic[attacking]{player};
and out-of possession \tikz\pic[defending]{player};

in team colors \tikz\pic[bronze,shorts=green]{player};

and in the same spot \tikz{\draw pic[grey]{player}  pic[away]{player};}

\subsection{Interactions}
% -----------------------------------------------------------------
styles for passing, dribbling, running

\begin{tabular}{cl}
\tikz\draw[pass] (0,0) -- (6,0); & pass\\
\tikz\draw[carry] (0,0) -- (6,0); & dribble\\
\tikz\draw[run] (0,0) -- (6,0); & run\\
\end{tabular}

\subsection{Placing Players and the Ball}\label{sec:players}
% -----------------------------------------------------------------
home and away, attacking and defending, numbered and labeled,
scaled and colored

{\tikzset{playerscale=1,gray}\sf
\begin{tabular}{cl@{\qquad}cl}
\tikz\pic[home]{player}; & \cs{pic[home]\{player\}} &
\tikz\pic[home,number=7]{player}; & \cs{pic[home,number=7]\{player\}} \\
\tikz\pic[away]{player}; & \cs{pic[away]\{player\}} &
	\tikz\pic[away,number=11]{player}; & \cs{pic[away,number=11]\{player\}} \\
\tikz\pic[home,shorts=green]{player}; & \cs{pic[home,shorts=green]\{player\}} &
\tikz\pic[home,shorts=green,number=11]{player}; & \cs{pic[home,shorts=green,number=7]\{player\}} \\
\tikz\pic[away,shorts=purple]{player}; & \cs{pic[away,shorts=purple]\{player\}} &
\tikz\pic[away,shorts=purple,number=10]{player}; & \cs{pic[away,shorts=purple,number=11]\{player\}} \\
\end{tabular}}

\tikz[playerscale=4]{%
 \draw pic(P)[home,grey]{player} +(20,0) pic(Q)[black,away,rotate=20]{player};
 \draw[red,thick] (P) -- (Q);
 \fill (Q.east) circle (0.5) node[right,rotate=20]{Q.east};
}

\tikz[playerscale=2]{%
 \draw pic(P)[home,grey]{player} +(20,0) pic(Q)[black,away]{player};
 \draw[red,thick] (P) -- (Q);
}

\tikz[playerscale=1]{%
 \draw pic(P)[home,grey]{player} +(10,-10) pic(Q)[black,away]{player};
 \draw[red,thick] (P) -- (Q);
}

player pictures?

\subsection{Routing Passes, Dribbles, and Runs}\label{sec:interactions}
% -----------------------------------------------------------------
straight, bend, concatenated, labeled, 


\subsection{Annotations}\label{sec:annotations}
% -----------------------------------------------------------------
areas, comments

training equipment?

\section{Animations}
% =================================================================

For now, example use of regular \TikZ\ features.
Macros for moving players planned.



\section{Playing out from the Back}
% =================================================================

\begin{tikzpicture}[pitchunit=1.5mm,>=stealth]

% -- pitch, on top of grass lines and tactical zones (default zones)
\pic[gray]{pitchgrass};
\pic[gray!50,dashed]{pitchzones};
% \pic[green,dashed]{gridzones={6}{3}};
\tikzset{pitchunit}
\pic[white]{pitch};

% -- players 

% cross to a #9
\begin{scope}[home,attacking]
\draw (-22,85) pic(LF)[possessor]{player};
\draw (10,90) pic(CF)[number=9]{player};
\draw[pass] (LF) to[bend left=12] (CF);
\end{scope}

% some defenders
\draw foreach \x/\y in {-20/90,-10/80,4/97} {(\x,\y) pic[away,defending]{player}};

\draw (-27,20) pic(S1)[possessor]{player} (-25,55) pic(S2)[shorts=yellow]{player};
\draw (-17,25) pic(T1)[number=7]{player}  (-15,40) pic(T2)[number=8,shorts=green]{player};
\draw[pass] (S1) -- (T1);
\draw[run]  (S1) to[bend left=10] ($(S1)!0.5!(S2)$) to[bend right=20] (S2);
\draw[carry](T1) -- (T2);
\draw[pass] (T2) -- (S2);

\begin{scope}[away,shorts=white]
\draw (-25,23) pic{player}
      (-12,27) pic[number=11]{player} node[above right]{name}; 
\end{scope}

\end{tikzpicture}

% =================================================================
\end{document}

% -- graveyard, but with goal cs example
\draw (goal cs:h=17,v=22,team=away) \player{RB}{2};
\GoalCS{away}{defending}
\draw (goal cs:h=17,v=24) \player{RB}{3};
\draw (goal cs:h=17,v=-22,team=away,phase=attacking) \player{RB}{4};
\draw (goal cs:h=19,v=24) \player{RB}{5};
